\documentclass{article}
\usepackage{amsthm}
\usepackage{amsmath}
\usepackage{amssymb}
\usepackage{stmaryrd}
\usepackage{hyperref}

\newtheorem{theorem}{Theorem}%[section]
\newtheorem{lemma}[theorem]{Lemma}
\newtheorem{proposition}[theorem]{Proposition}
\newtheorem{corollary}{Corollary}
\newtheorem{definition}{Definition}%[section]
\newtheorem{conjecture}{Conjecture}%[section]
\newtheorem{example}{Example}%[section]
\newtheorem{remark}{Remark}

\newcommand{\Reals}[0]{\mathbb{R}}

\title{Notes on Topology}

\author{Jeremy G. Siek}

\begin{document}
\maketitle

\begin{definition}[Continuous]
  A function $f : A \to \Reals$, with $A \subseteq \Reals$,
  is continuous at $c \in A$ if for every $\epsilon > 0$,
  there exists $\delta > 0$ such that
  \[
    |x - c| < \delta \land x \in A \implies |f(x) - f(c)| < \epsilon
  \]
  A function $f$ is continuous on a set $B \subseteq A$ if it is
  continuous on every element in B.
\end{definition}

\begin{definition}[Ball]
  $B_r(x) = (x-r,x+r)$
\end{definition}

\begin{definition}[Neighborhood]
  A set $U \subseteq \Reals$ is a neighborhood of $x \in \Reals$
  if there is a $\delta > 0$ such that $B_\delta(x) \subseteq U$.
\end{definition}


\begin{definition}[N-Continuous]
  A function $f : A \to \Reals$, with $A \subseteq \Reals$,
  is n-continuous at $c \in A$ if for every neighborhood
  $V$ of $f(c)$ there exists a neighborhood $U$ of $c$
  such that
  \[
    x \in A \cap U \implies f(x) \in V
  \]
\end{definition}

\begin{proposition}
  Continuous and n-continuous are equivalent.
\end{proposition}
\begin{proof}\ 
   \begin{enumerate}
   \item (continuous $\implies$ n-continuous)
     Suppose $f$ is continuous at $c$.
     Let $V$ be a neighborhood of $f(c)$.
     So there exists a $\epsilon > 0$ s.t. 
     \begin{equation} \label{eq:ball-V-1}
     B_\epsilon(f(c)) \subseteq V
     \end{equation}
     Because $f : A \to \Reals$ is continuous at $c$, 
     there exists $\delta > 0$ s.t.
     \begin{equation} \label{eq:ed-cont-1}
     |x - c| < \delta \land x \in A \implies |f(x) - f(c)| < \epsilon
     \end{equation}
     Choose $U=B_\delta(c)$. nts. 
     \[
     x \in A \cap U \implies f(x) \in V
     \]
     We assume $x \in A \cap U$. Then by \eqref{eq:ed-cont-1} we have
     $|f(x) - f(c)| < \epsilon$. So $f(x) \in B_\epsilon(f(c))$
     and so also $f(x) \in V$ by \eqref{eq:ball-V-1}.
     Therefore $f$ is n-continuous.

   \item (n-continuous $\implies$ continuous) Suppose $f$ is
     n-continouous at $c$ and $\epsilon > 0$. Note that
     $B_\epsilon(f(c))$ is a neighborhood of $f(c)$. Then because $f$
     n-continuous there is a neighborhood $U$ of $c$ such that
     \begin{equation}\label{eq:xU-fx-Be}
       x \in A \cap U \implies f(x) \in B_\epsilon(f(c))
     \end{equation}
     Because $U$ is a neighborhood of $c$, there is a $\delta > 0$
     such that $B_\delta(c) \subseteq U$. We need to show that
     \[
     |x - c| < \delta \land x \in A \implies |f(x) - f(c)| < \epsilon
     \]
     Assume $|x - c| < \delta$ and $x \in A$.
     We have $x \in U$, then by \eqref{eq:xU-fx-Be} obtain
     $f(x) \in B_\epsilon(f(c))$. From this we conclude that
     $|f(x) - f(c)| < \epsilon$.

   \end{enumerate}
\end{proof}


\end{document}
